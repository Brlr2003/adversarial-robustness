% Main Thesis Document
% Epoka University MSc Thesis Template
% Topic: Adversarial Attacks on Deep Neural Networks
% Author: Omar Alsafarti

\documentclass{epoka}

% Additional packages you might need
\usepackage{amsmath}
\usepackage{amssymb}
\usepackage{graphicx}
\usepackage{cite}
\usepackage{url}
\usepackage{booktabs}
\usepackage{multirow}
\usepackage{longtable}
\usepackage{algorithm}
\usepackage{algorithmic}
\usepackage{listings}
\usepackage{xcolor}
\usepackage{subcaption}
\usepackage{hyperref}
\usepackage{tcolorbox}  % For colored boxes that work with paragraphs

% ============================================================
% HIGHLIGHTING FOR INCOMPLETE SECTIONS
% Yellow highlight = Content you need to write/complete
% ============================================================
\definecolor{todocolor}{RGB}{255,255,150}
\definecolor{todotextcolor}{RGB}{180,100,0}

% Simple inline TODO - just colored text in brackets
\newcommand{\TODO}[1]{\textcolor{todotextcolor}{\textbf{[TODO: #1]}}}

% Block placeholder - using tcolorbox for multi-line content
\newtcolorbox{placeholderbox}{
    colback=todocolor,
    colframe=todotextcolor,
    boxrule=1pt,
    arc=3pt,
    left=5pt,
    right=5pt,
    top=5pt,
    bottom=5pt
}

% Simple placeholder command for single-line use (if needed)
% Use \begin{placeholderbox}...\end{placeholderbox} for multi-line content
% ============================================================

% Hyperref setup
\hypersetup{
    colorlinks=true,
    linkcolor=black,
    filecolor=black,
    citecolor=black,
    urlcolor=blue,
    pdftitle={Adversarial Attacks on Deep Neural Networks},
    pdfauthor={Omar Alsafarti},
}

% Code listing setup
\lstset{
    basicstyle=\ttfamily\small,
    breaklines=true,
    frame=single,
    numbers=left,
    numberstyle=\tiny,
    captionpos=b,
    language=Python
}

% Include your thesis metadata
% Thesis Metadata
% Fill in your personal information and thesis details here

% Thesis type
\MSc  % For Master of Science

% Author information
\author{Omar}{Alsafarti}
\studentid{42012418}

% Thesis titles
\title{ADVERSARIAL ATTACKS ON DEEP NEURAL NETWORKS: GENERATION, EVALUATION, AND ROBUST TRAINING}
\titleenglish{ADVERSARIAL ATTACKS ON DEEP NEURAL NETWORKS: GENERATION, EVALUATION, AND ROBUST TRAINING}
\titlealbanian{SULMET KUNDËRSHTARE NË RRJETET NEURALE TË THELLA: GJENERIMI, VLERËSIMI DHE TRAJNIMI I QËNDRUESHËM}

% Dates
\date{June 2026}
\dateenglish{June 2026}

% Supervisor
\supervisor{Assoc. Prof. Dr. Arban Uka}
\supervisorenglish{Assoc. Prof. Dr. Arban Uka}

% Department and Faculty
\department{Computer Engineering}
\departmentenglish{Computer Engineering}
\faculty{Faculty of Architecture and Engineering}
\facultyenglish{Faculty of Architecture and Engineering}

% Head of Department (for approval page)
\headofdepartment{Dr. Florenc Skuka} 

% Committee members (for approval page)
\committee
    {Assoc. Prof. Dr. Arban Uka}  % Chair/Supervisor
    {Assoc. Prof. Dr. [Member Two]}  % TODO: Add committee member
    {Asst. Prof. Dr. [Member Three]}  % TODO: Add committee member
    {}  % Leave empty if not applicable
    {}  % Leave empty if not applicable

% Abstract (English)
\abstract{
Deep learning models achieve state-of-the-art performance on tasks such as image classification, yet they are notoriously vulnerable to adversarial examples---inputs that have been perturbed by small, often imperceptible changes which cause misclassification with high confidence. This vulnerability raises serious security and safety concerns for systems deployed in autonomous driving, biometrics, and cybersecurity. Building on foundational work, adversarial attacks have been formalized as optimization problems that exploit gradients of deep neural networks to construct worst-case perturbations under various norm constraints.

This thesis focuses on the practical study of adversarial attacks against image classification models. A ResNet-18 convolutional neural network is trained on the CIFAR-10 benchmark dataset, and three white-box attacks (FGSM, PGD, DeepFool) are implemented and compared. The standard model achieves 94.1\% clean accuracy but drops to 1.2\% under 20-step PGD attack at $\epsilon = 8/255$. PGD-based adversarial training is then applied, producing a robust model that maintains 45.1\% accuracy under the same attack at the cost of a 10.2 percentage point reduction in clean accuracy. DeepFool analysis reveals that the standard model's decision boundaries are on average only 0.248 $L_2$ distance from the data points, while adversarial training increases this distance to 0.892. These results confirm the severity of adversarial vulnerability in deep neural networks and validate adversarial training as an effective, practical defense mechanism.
}

% Keywords (English)
\keywordsenglish{Adversarial Examples, Deep Neural Networks, FGSM, PGD, DeepFool, Adversarial Training, Robustness}

% Abstrakt (Albanian)
\abstrakt{
Modelet e mësimit të thellë arrijnë performancë të jashtëzakonshme në detyra si klasifikimi i imazheve, megjithatë ato janë të njohura për cenueshmërinë ndaj shembujve kundërshtarë---inpute që janë ndryshuar me ndryshime të vogla, shpesh të padukshme, të cilat shkaktojnë klasifikim të gabuar me besueshmëri të lartë. Kjo cenueshmëri ngre shqetësime serioze për sigurinë e sistemeve të përdorura në drejtimin autonom, biometrikë dhe sigurinë kibernetike.

Kjo tezë fokusohet në studimin praktik të sulmeve kundërshtare kundër modeleve të klasifikimit të imazheve. Një rrjet nervor konvolucionar ResNet-18 trajnohet në grupin e të dhënave CIFAR-10 dhe tre sulme white-box (FGSM, PGD, DeepFool) implementohen dhe krahasohen. Modeli standard arrin saktësi 94.1\% në të dhëna të pastra por bie në 1.2\% nën sulmin PGD me 20 hapa. Trajnimi kundërshtar i bazuar në PGD aplikohet më pas, duke prodhuar një model të qëndrueshëm që ruan 45.1\% saktësi nën të njëjtin sulm, me koston e një rënieje prej 10.2 pikësh përqindjeje në saktësinë e pastër. Këto rezultate konfirmojnë ashpërsinë e cenueshmërisë kundërshtare dhe vërtetojnë trajnimin kundërshtar si një mekanizëm mbrojtës efektiv.
}

% Keywords (Albanian)
\keywords{Shembuj Kundërshtarë, Rrjete Neurale të Thella, FGSM, PGD, DeepFool, Trajnim Kundërshtar, Qëndrueshmëri}

% Acknowledgments
\acknowledgments{
I would like to express my sincere gratitude to my supervisor, Assoc. Prof. Dr. Arban Uka, for his invaluable guidance, support, and encouragement throughout my research. His expertise in the field and constructive feedback were instrumental in shaping this work.

I am also grateful to my family for their continuous support and understanding during my studies, and to my friends and colleagues at Epoka University for the stimulating academic environment.

Finally, I would like to thank Epoka University for providing me with the opportunity to pursue my graduate studies and for the resources that made this research possible.
}

% Dedication (optional - uncomment if needed)
% \dedication{
%     To my family...
% }

% Abbreviations
\abbreviations{
\begin{tabular}{ll}
AE & Adversarial Example \\
BIM & Basic Iterative Method \\
C\&W & Carlini-Wagner Attack \\
CNN & Convolutional Neural Network \\
DNN & Deep Neural Network \\
FGSM & Fast Gradient Sign Method \\
GPU & Graphics Processing Unit \\
MPS & Metal Performance Shaders \\
PGD & Projected Gradient Descent \\
PGD-AT & PGD-based Adversarial Training \\
ResNet & Residual Network \\
SGD & Stochastic Gradient Descent \\
UAP & Universal Adversarial Perturbation \\
\end{tabular}
}

% Optional: Disable table or figure lists if not needed
% \NoTableList
% \NoFigureList

\begin{document}

% Generate front matter (cover page, abstract, TOC, etc.)
\makefrontmatter

% Include chapters
\chapter{INTRODUCTION}
\label{ch:introduction}

\section{Introduction}

Over the past decade, deep neural networks (DNNs) have become the go-to approach for tasks ranging from computer vision to speech recognition. They work remarkably well in practice, but they come with a surprising flaw. If someone adds a tiny amount of carefully chosen noise to an image, noise so small that a person would never notice it, the network can suddenly produce a completely wrong prediction, and it will do so with high confidence~\cite{szegedy2014intriguing,goodfellow2015explaining}. These manipulated inputs are called adversarial examples.

What makes this problem particularly worrying is that it is not limited to laboratory settings. Researchers have shown that adversarial examples can fool models in the physical world too, for instance through printed images held up to a camera or subtly altered road signs that confuse self-driving car systems~\cite{kurakin2016adversarialphysical}. When you consider that deep learning is being used more and more in areas where mistakes can be dangerous (think autonomous driving, medical imaging, or security screening), the stakes become clear. This has pushed a lot of researchers to look into both how these attacks work and how we might defend against them~\cite{yuan2019adversarial,ren2020adversarial,chakraborty2018survey,zhou2022adversarial,macas2024adversarial}.

\section{Background and Motivation}

The story of adversarial examples really began with the work of Szegedy et al.~\cite{szegedy2014intriguing}. They were the first to show, in a systematic way, that even well-trained neural networks could be tricked by perturbations that are barely visible. Shortly after, Goodfellow et al.~\cite{goodfellow2015explaining} offered an explanation for why this happens. They argued that it has to do with the linear behavior of networks in high-dimensional spaces, and they introduced the Fast Gradient Sign Method (FGSM), a simple but effective way to generate adversarial examples using gradient information.

My motivation for this research comes from a straightforward observation: if we are going to trust deep learning models in applications like autonomous vehicles, medical diagnosis, or security systems, we need to understand how fragile they really are. Studying how adversarial attacks work is, in my view, the necessary first step before we can build models that are genuinely robust.

% TODO: Consider adding a figure showing adversarial example concept
% \begin{figure}[htbp]
%     \centering
%     \includegraphics[width=0.8\textwidth]{figures/adversarial_example.png}
%     \caption{Illustration of an adversarial example: a small perturbation causes misclassification.}
%     \label{fig:adversarial_example}
% \end{figure}

\section{Problem Statement}

I am working with standard image classification models, specifically convolutional neural networks trained on MNIST and CIFAR-10, and this thesis tries to answer three questions:

\begin{enumerate}
    \item How vulnerable are these models to common white-box adversarial attacks (FGSM, BIM/PGD, DeepFool) when we vary the perturbation norms and budgets?
    \item How well does adversarial training (particularly PGD-based) improve robustness, and how much clean accuracy do we lose in the process?
    \item How do the different attacks compare when we look at success rate, perturbation size, and computational cost?
\end{enumerate}

I focus specifically on evasion attacks at test time and use image classification as a concrete setting for studying these questions about neural network robustness.

\section{Research Objectives}

The goal of this research is to connect the theoretical side of adversarial attacks with hands-on implementation and experimentation. More specifically, I set out to:

\begin{itemize}
    \item Implement and evaluate several well-known adversarial attacks (FGSM, BIM/PGD, DeepFool) on standard benchmark datasets.
    \item Measure how vulnerable convolutional neural networks are to these attacks under different perturbation constraints.
    \item Apply adversarial training as a defense and see how much it actually helps in terms of model robustness.
    \item Look at the trade-offs between clean accuracy and robust accuracy across different configurations.
\end{itemize}

\section{Contributions}

The main contributions of this thesis are:

\begin{enumerate}
    \item An empirical comparison of three classic adversarial attacks on a standard deep learning model, showing how they differ in effectiveness and computational cost.
    \item A practical demonstration of how adversarial training changes a model's robustness profile, including the trade-offs involved.
    \item A structured and reproducible experimental framework that ties together the implementation with the theoretical background covered in the literature.
\end{enumerate}

\section{Thesis Organization}

The rest of this thesis is organized as follows:

\textbf{Chapter 2} reviews the existing literature on adversarial examples, covering the theoretical foundations, the different types of attacks, and the defenses that have been proposed so far.

\textbf{Chapter 3} explains the methodology I used, including the mathematical formulation of each attack, the datasets and model architectures, and how I designed the experiments.

\textbf{Chapter 4} goes into the implementation details: how the attack algorithms and training procedures were coded, the software architecture, and the development environment.

\textbf{Chapter 5} presents the experimental results, with numbers on attack success rates, perturbation sizes, and how well adversarial training works as a defense.

\textbf{Chapter 6} wraps things up with conclusions, a discussion of limitations, and ideas for future work.

\chapter{LITERATURE REVIEW}
\label{ch:literature}

\section{Introduction}

In this chapter we review the existing literature on adversarial examples for deep neural networks. We cover the theoretical foundations, the main attack methods, and the proposed defenses.

\section{Theoretical Background}

\subsection{Adversarial Examples and Threat Models}

Szegedy et al.~\cite{szegedy2014intriguing} first demonstrated that neural networks with high test accuracy can be fooled by adding small, imperceptible perturbations to their inputs. Prior to this result, it was generally assumed that these networks were learning meaningful features of the data. The finding that small perturbations could cause confident misclassification suggested otherwise.

Goodfellow et al.~\cite{goodfellow2015explaining} proposed that the vulnerability arises from the linear behavior of neural networks in high-dimensional spaces. Even though the networks contain nonlinear activation functions, a perturbation that is small in each dimension can accumulate across many dimensions and produce a large change in the output. To demonstrate this, they introduced FGSM --- a method that generates adversarial examples using a single gradient step. The method is simple, and the fact that it works well indicates that the vulnerability is systematic rather than incidental.

\subsection{Threat Model Taxonomy}

Adversarial attacks are typically categorized along three axes:

\begin{itemize}
    \item \textbf{Attacker knowledge.} In the \emph{white-box} setting, the attacker has full access to the model, including architecture, weights, and gradients. In the \emph{black-box} setting, the attacker can only query the model and observe outputs.

    \item \textbf{Attacker goal.} \emph{Untargeted} attacks aim to cause any misclassification. \emph{Targeted} attacks aim to produce a specific incorrect prediction.

    \item \textbf{Perturbation constraint.} The perturbation is typically bounded under an $L_p$ norm. $L_\infty$ bounds the maximum change per pixel, $L_2$ bounds the Euclidean distance, and $L_0$ bounds the number of modified pixels.
\end{itemize}

We refer the reader to Yuan et al.~\cite{yuan2019adversarial} and Ren et al.~\cite{ren2020adversarial} for thorough surveys of threat models.

\section{Related Work}

\subsection{Gradient-Based Attacks}

\subsubsection{Fast Gradient Sign Method (FGSM)}

FGSM~\cite{goodfellow2015explaining} computes the gradient of the loss with respect to the input, takes the sign, and scales by the perturbation budget $\epsilon$:

\begin{equation}
    x^{adv} = x + \epsilon \cdot \text{sign}(\nabla_x J(\theta, x, y))
    \label{eq:fgsm}
\end{equation}

It requires one forward pass and one backward pass. Because it takes only a single step, FGSM does not necessarily find the strongest adversarial example within the $\epsilon$-ball.

\subsubsection{Basic Iterative Method (BIM) / Projected Gradient Descent (PGD)}

Kurakin et al.~\cite{kurakin2016adversarialscale} extended FGSM by applying it iteratively with a smaller step size:

\begin{equation}
    x^{adv}_{t+1} = \text{Clip}_{x,\epsilon}\{x^{adv}_t + \alpha \cdot \text{sign}(\nabla_x J(\theta, x^{adv}_t, y))\}
    \label{eq:bim}
\end{equation}

Madry et al.~\cite{madry2018towards} formalized this as projected gradient descent on the inner maximization problem. The main modification was initializing from a random point within the $\epsilon$-ball, which helps the attack explore the loss landscape more effectively. PGD has since become the standard first-order attack for evaluating adversarial robustness. If a model withstands PGD, this is taken as evidence of robustness against first-order adversaries.

\subsection{Optimization-Based Attacks}

\subsubsection{DeepFool}

Moosavi-Dezfooli et al.~\cite{moosavidezfooli2016deepfool} asked a different question: what is the \emph{smallest} perturbation that changes the prediction? DeepFool answers this by iteratively linearizing the classifier and computing the distance to the nearest decision boundary. The resulting perturbations are much smaller than those of FGSM, yet they still cause misclassification. We remark that DeepFool provides geometric information about the model --- specifically, the distance from each data point to the nearest decision boundary.

\subsubsection{Carlini-Wagner (C\&W) Attack}

Carlini and Wagner~\cite{carlini2017towards} formulated adversarial example generation as an optimization problem:

\begin{equation}
    \min_\delta \|\delta\|_p + c \cdot f(x + \delta)
    \label{eq:cw}
\end{equation}

where $f$ is designed such that $f(x') \leq 0$ implies misclassification. The C\&W attack demonstrated that several defenses previously claimed to be effective were in fact broken. It remains one of the strongest attacks, though it is computationally expensive compared to FGSM or PGD.

\subsubsection{Universal Adversarial Perturbations}

Moosavi-Dezfooli et al.~\cite{moosavidezfooli2017universal} showed that a single, fixed perturbation can fool a classifier on the majority of inputs. This is not an input-specific perturbation --- the same noise pattern is applied to all images. The existence of such universal perturbations suggests that the vulnerability is structural rather than tied to individual inputs.

\subsection{Black-Box Attacks and Transferability}

Papernot et al.~\cite{papernot2017practical} showed that white-box access is not required. An attacker can train a substitute model that approximates the target, generate adversarial examples against the substitute, and transfer them to the target model. This transfer attack works well in practice, which means an attacker does not need knowledge of the target model's internals.

\section{Defense Mechanisms}

\subsection{Adversarial Training}

The most direct defense is to augment training with adversarial examples~\cite{goodfellow2015explaining,kurakin2016adversarialscale,madry2018towards}. This amounts to solving a min-max optimization problem:

\begin{equation}
    \min_\theta \mathbb{E}_{(x,y) \sim \mathcal{D}} \left[ \max_{\|\delta\| \leq \epsilon} J(\theta, x + \delta, y) \right]
    \label{eq:adv_training}
\end{equation}

Madry et al.~\cite{madry2018towards} showed that using PGD for the inner maximization produces substantially better robustness than using FGSM. The cost is computational: running a multi-step attack at each training iteration makes training roughly 5--10$\times$ slower. Shafahi et al.~\cite{shafahi2019adversarialfree} addressed this by reusing gradient computations, reducing the overhead of adversarial training.

\subsection{Gradient Masking and Obfuscation}

A number of defenses attempted to prevent attacks by making gradients uninformative. Athalye et al.~\cite{athalye2018obfuscated} showed that nearly all such defenses could be circumvented. They developed BPDA (backward pass differentiable approximation) and related techniques to recover useful gradients despite the defense. Many defenses that had been reported as effective turned out to provide little actual robustness.

\subsection{Other Defense Approaches}

Several other defense strategies have been explored:

\begin{itemize}
    \item \textbf{Input preprocessing:} Applying transformations such as JPEG compression or denoising before classification, with the goal of removing adversarial perturbations.
    \item \textbf{Certified defenses:} Providing formal guarantees that the prediction will not change within a given perturbation radius. These guarantees come at the cost of reduced accuracy.
    \item \textbf{Randomized smoothing:} Classifying by majority vote over noisy copies of the input, which yields certifiable robustness in a probabilistic sense.
\end{itemize}

We refer the reader to the surveys by Ren et al.~\cite{ren2020adversarial}, Chakraborty et al.~\cite{chakraborty2018survey}, Zhou et al.~\cite{zhou2022adversarial}, and Macas et al.~\cite{macas2024adversarial} for further detail.

\section{Research Gap}

We identify several gaps in the existing literature:

\begin{enumerate}
    \item Most works focus on a single attack or a single defense. We found it difficult to locate a study that compares FGSM, PGD, and DeepFool on the same model under the same evaluation protocol. Different papers use different configurations, which makes cross-paper comparison unreliable.

    \item The accuracy--robustness trade-off is hard to assess from the literature because experimental setups vary: different $\epsilon$ values, different numbers of PGD steps, different architectures. It is often unclear whether differences in reported numbers come from the method or the setup.

    \item Implementation details are frequently underspecified. Papers report hyperparameters but omit details that matter in practice --- for example, the handling of batch normalization during adversarial training, which we found to have a substantial effect on the final model.
\end{enumerate}

This thesis addresses these gaps by providing a single study with a unified codebase, consistent evaluation, and sufficient implementation detail for reproducibility.

\section{Summary}

The literature indicates that adversarial vulnerability is a fundamental property of current deep learning models rather than an artifact that can be patched. Attacks range from single-step methods (FGSM) to full optimization procedures (C\&W). On the defense side, adversarial training~\cite{madry2018towards} remains the most reliable approach, despite its cost in training time and clean accuracy. The remainder of this thesis builds on these foundations with our own implementations and experiments.


\chapter{METHODOLOGY}
\label{ch:methodology}

\section{Introduction}

This chapter describes the experimental setup. We cover the datasets, model architectures, attack formulations, defense method, and evaluation metrics.

\section{Research Design}

We structured the work into four phases:

\begin{enumerate}
    \item \textbf{Model Training:} Train baseline CNNs on the benchmark datasets and establish clean accuracy.

    \item \textbf{Attack Implementation:} Implement FGSM, BIM/PGD, and DeepFool from scratch and evaluate them against the baselines.

    \item \textbf{Defense Application:} Apply PGD-based adversarial training and measure the resulting robustness.

    \item \textbf{Comparative Analysis:} Compare all attacks and models under identical evaluation conditions.
\end{enumerate}

\section{Datasets}

\subsection{MNIST}

MNIST~\cite{lecun1998mnist} contains 70,000 grayscale images of handwritten digits (0--9), split into 60,000 training and 10,000 test images at $28 \times 28$ resolution. Standard models achieve above 99\% accuracy on this dataset. We used MNIST primarily for validating our attack implementations before moving to the more challenging CIFAR-10.

\subsection{CIFAR-10}

CIFAR-10 contains 60,000 color images across 10 classes, with 50,000 training and 10,000 test images at $32 \times 32$ resolution with 3 color channels. This dataset is substantially harder than MNIST and requires a capable model architecture to achieve good accuracy. We chose CIFAR-10 as our primary benchmark because it provides a non-trivial classification problem on which to evaluate attacks and defenses.

% TODO: Add figure showing sample images from datasets
% \begin{figure}[htbp]
%     \centering
%     \includegraphics[width=0.8\textwidth]{figures/datasets.png}
%     \caption{Sample images from MNIST (left) and CIFAR-10 (right) datasets.}
%     \label{fig:datasets}
% \end{figure}

\section{Model Architectures}

\subsection{MNIST Model}

For MNIST we used a small CNN:

\begin{itemize}
    \item Conv2D (32 filters, 3$\times$3) $\rightarrow$ ReLU $\rightarrow$ Conv2D (64 filters, 3$\times$3) $\rightarrow$ ReLU
    \item MaxPooling (2$\times$2) $\rightarrow$ Dropout (0.25)
    \item Flatten $\rightarrow$ Dense (128) $\rightarrow$ ReLU $\rightarrow$ Dropout (0.5)
    \item Dense (10) $\rightarrow$ Softmax
\end{itemize}

This is a standard architecture for MNIST that achieves the accuracy needed for our purposes.

\subsection{CIFAR-10 Model}

For CIFAR-10 we used ResNet-18, modified for $32 \times 32$ inputs. The standard ResNet-18 architecture is designed for $224 \times 224$ images, so we made several changes to the first convolutional layer and pooling. The specific modifications are described in Chapter~\ref{ch:implementation}.

\section{Attack Implementations}

\subsection{Fast Gradient Sign Method (FGSM)}

FGSM is given by Equation~\ref{eq:fgsm_method}:

\begin{equation}
    x^{adv} = x + \epsilon \cdot \text{sign}(\nabla_x J(\theta, x, y))
    \label{eq:fgsm_method}
\end{equation}

We evaluated FGSM across multiple $\epsilon$ values. For MNIST: $\epsilon \in \{0.1, 0.2, 0.3\}$. For CIFAR-10: $\epsilon \in \{2/255, 4/255, 8/255\}$. This sweep allows us to characterize how attack success scales with perturbation budget.

\subsection{Basic Iterative Method / Projected Gradient Descent (BIM/PGD)}

PGD applies FGSM iteratively with a smaller step size and projects back into the constraint set after each step:

\begin{equation}
    x^{adv}_{t+1} = \Pi_{x,\epsilon}\left(x^{adv}_t + \alpha \cdot \text{sign}(\nabla_x J(\theta, x^{adv}_t, y))\right)
    \label{eq:pgd_method}
\end{equation}

Here $\Pi_{x,\epsilon}$ denotes projection onto the $\epsilon$-ball. We varied the following parameters:
\begin{itemize}
    \item Step size $\alpha$: $\epsilon/10$ or $\epsilon/4$
    \item Number of iterations: 10, 20, or 40
    \item Initialization: random start within the $\epsilon$-ball (PGD) versus starting from the clean input (BIM)
\end{itemize}

\subsection{DeepFool}

DeepFool~\cite{moosavidezfooli2016deepfool} differs from the above methods in that it does not operate within a fixed $\epsilon$ budget. Instead, it finds the \emph{smallest} perturbation that changes the model's prediction by iteratively linearizing the classifier and computing the distance to the nearest decision boundary. The full algorithm is described in Chapter~\ref{ch:implementation}.

\subsection{(Optional) Carlini-Wagner Attack}

We initially planned to include the C\&W attack~\cite{carlini2017towards}:

\begin{equation}
    \min_\delta \|\delta\|_2 + c \cdot f(x + \delta)
\end{equation}

where $f$ is designed so that $f(x') \leq 0$ implies misclassification. However, the computational cost proved prohibitive on our hardware, and we were unable to run it within a reasonable time frame.


\section{Defense: Adversarial Training}

\subsection{FGSM Adversarial Training}

For MNIST we used FGSM-based adversarial training following Goodfellow et al.~\cite{goodfellow2015explaining}:

\begin{enumerate}
    \item Generate FGSM adversarial examples from each mini-batch
    \item Mix adversarial and clean examples at a 50/50 ratio
    \item Train on the combined batch
\end{enumerate}

\subsection{PGD Adversarial Training}

For CIFAR-10 we used PGD-based adversarial training following Madry et al.~\cite{madry2018towards}:

\begin{enumerate}
    \item For each mini-batch, generate adversarial examples using multi-step PGD
    \item Train on the adversarial examples
    \item Perturbation budget: $\epsilon = 8/255$ under $L_\infty$
\end{enumerate}

The full set of hyperparameters is given in Chapter~\ref{ch:results}.

\section{Evaluation Metrics}

We report the following metrics for each experiment:

\subsection{Accuracy Metrics}

\begin{itemize}
    \item \textbf{Clean Accuracy:} Accuracy on unperturbed test images.
    \item \textbf{Robust Accuracy:} Accuracy on adversarially perturbed test images, measured at each $\epsilon$.
    \item \textbf{Attack Success Rate:} The fraction of correctly classified clean images that the attack successfully causes to be misclassified.
\end{itemize}

\subsection{Perturbation Metrics}

\begin{itemize}
    \item \textbf{Average $L_2$ norm:} Mean Euclidean distance between clean and adversarial images.
    \item \textbf{Average $L_\infty$ norm:} Mean maximum per-pixel change across the test set.
\end{itemize}

\subsection{Comparison Table}

Table~\ref{tab:methodology_comparison} summarizes the experimental configurations.

\begin{table}[htbp]
    \centering
    \caption{Summary of experimental configurations}
    \label{tab:methodology_comparison}
    \begin{tabular}{llll}
        \toprule
        \textbf{Dataset} & \textbf{Model} & \textbf{Attacks} & \textbf{Defense} \\
        \midrule
        MNIST & Small CNN & FGSM, PGD, DeepFool & FGSM-AT \\
        CIFAR-10 & ResNet-18/Custom & FGSM, PGD, DeepFool & PGD-AT \\
        \bottomrule
    \end{tabular}
\end{table}

\section{Tools and Technologies}

\begin{itemize}
    \item \textbf{Python 3.x}
    \item \textbf{PyTorch:} all model training and attack implementations
    \item \textbf{Custom implementations:} no external adversarial ML libraries
    \item \textbf{NumPy, Matplotlib:} data handling and visualization
    \item \textbf{Hardware:} Apple M4 Pro with MPS (Metal Performance Shaders) backend
\end{itemize}

\section{Summary}

This chapter described the experimental design: standard benchmark datasets, a well-known model architecture, three attacks implemented from scratch, PGD adversarial training as the defense, and a set of metrics for evaluation. The next chapter covers the implementation.


\chapter{IMPLEMENTATION}
\label{ch:implementation}

\section{Introduction}

This chapter describes the implementation of the adversarial attack algorithms, defense mechanisms, and the complete experimental framework developed for this research. The implementation follows a modular software architecture built entirely in PyTorch, with custom implementations of all three attack algorithms (FGSM, PGD, DeepFool), both standard and adversarial training pipelines, and an interactive demonstration application deployed on HuggingFace Spaces.

The transition from methodology to functional implementation required careful attention to computational efficiency, numerical stability, and reproducibility. All experiments were tracked using MLflow for systematic comparison of results across different configurations.

\section{System Architecture}

The implementation follows a modular architecture designed to separate concerns and facilitate systematic experimentation. The system consists of five primary modules:

The \textbf{Model Module} contains the ResNet-18 architecture adapted for CIFAR-10. The \textbf{Attack Module} provides independent implementations of each adversarial attack algorithm with a consistent interface. The \textbf{Training Module} implements both standard and adversarial training loops with experiment tracking. The \textbf{Evaluation Module} computes robustness metrics across multiple attack types and perturbation budgets. Finally, the \textbf{Frontend Module} provides an interactive Streamlit-based demonstration interface.

Each module operates independently, communicating through well-defined interfaces. This design allows attacks to be swapped or added without modifying the training or evaluation code, and enables the same evaluation pipeline to assess both standard and adversarially trained models.

\section{Implementation Details}

\subsection{Model Implementation}

For CIFAR-10, a ResNet-18 architecture~\cite{he2016deep} was implemented with modifications appropriate for the smaller input resolution. Standard ResNet-18 was designed for ImageNet images of size $224 \times 224$ pixels, while CIFAR-10 images are only $32 \times 32$ pixels. To prevent excessive spatial downsampling, two key modifications were made: the initial $7 \times 7$ convolution with stride 2 was replaced with a $3 \times 3$ convolution with stride 1 and padding 1, and the initial max pooling layer was removed entirely. These changes preserve spatial resolution in the early layers, which is critical for the small input dimensions.

The architecture consists of four residual layers, each containing two BasicBlock modules. Each BasicBlock performs two $3 \times 3$ convolutions with batch normalization and ReLU activation, connected by a skip (residual) connection. When the spatial dimensions are reduced (layers 2--4), a $1 \times 1$ convolution is used in the shortcut path to match dimensions. The channel dimensions progress from 64 to 128 to 256 to 512 across the four layers. After the final residual layer, adaptive average pooling reduces the spatial dimensions to $1 \times 1$, followed by a fully connected layer that produces 10 class logits. The total model contains approximately 11.17 million parameters.

Weights were initialized using Kaiming normal initialization for convolutional layers and constant initialization for batch normalization layers, following standard practice for ResNet architectures.

An important design decision was made regarding input normalization: images were kept in the $[0, 1]$ pixel range without applying channel-wise normalization (mean subtraction and standard deviation division). This choice was deliberate because adversarial perturbation budgets ($\epsilon$) are defined in pixel space. Applying normalization would require adjusting epsilon values to account for the scaling, adding unnecessary complexity. Keeping inputs in $[0, 1]$ ensures that $\epsilon = 8/255 \approx 0.031$ directly corresponds to a maximum per-pixel change of 8 intensity levels out of 255, which is the standard benchmark value in the adversarial robustness literature~\cite{madry2018towards}.

\subsection{Data Loading and Preprocessing}

The CIFAR-10 dataset was loaded using torchvision with the following preprocessing pipeline. For training, data augmentation was applied consisting of random cropping with 4-pixel padding and random horizontal flipping, followed by conversion to tensor format. For testing, only tensor conversion was applied. Data was loaded in batches of 128 using PyTorch's DataLoader with 4 worker processes and pinned memory for efficient GPU transfer.

\subsection{FGSM Attack Implementation}

The FGSM attack was implemented following the formulation of Goodfellow et al.~\cite{goodfellow2015explaining} as described in Algorithm~\ref{alg:fgsm}. The implementation accepts a batch of images and their true labels, computes the cross-entropy loss, performs backpropagation to obtain input gradients, and generates adversarial examples in a single step.

\begin{algorithm}[htbp]
\caption{Fast Gradient Sign Method (FGSM)}
\label{alg:fgsm}
\begin{algorithmic}[1]
\REQUIRE Input image $x$, true label $y$, model $f_\theta$, perturbation budget $\epsilon$
\ENSURE Adversarial example $x^{adv}$
\STATE Compute loss: $L = J(\theta, x, y)$
\STATE Compute gradient: $g = \nabla_x L$
\STATE Compute perturbation: $\delta = \epsilon \cdot \text{sign}(g)$
\STATE Generate adversarial example: $x^{adv} = \text{clip}(x + \delta, 0, 1)$
\RETURN $x^{adv}$
\end{algorithmic}
\end{algorithm}

The gradient is computed with respect to the input images by setting \texttt{requires\_grad = True} on the input tensor before the forward pass. After backpropagation, the sign of the gradient is taken element-wise and scaled by $\epsilon$. The resulting adversarial images are clamped to the valid pixel range $[0, 1]$ to ensure they remain valid images. The implementation operates on full batches for computational efficiency.

\subsection{PGD Attack Implementation}

The PGD attack was implemented following Madry et al.~\cite{madry2018towards} as an iterative extension of FGSM, described in Algorithm~\ref{alg:pgd}. PGD applies multiple smaller gradient steps while projecting the result back into the $L_\infty$ $\epsilon$-ball around the original input after each iteration.

\begin{algorithm}[htbp]
\caption{Projected Gradient Descent (PGD) Attack}
\label{alg:pgd}
\begin{algorithmic}[1]
\REQUIRE Input $x$, label $y$, model $f_\theta$, budget $\epsilon$, step size $\alpha$, iterations $T$
\ENSURE Adversarial example $x^{adv}$
\STATE Initialize: $x^{adv}_0 = x + \text{Uniform}(-\epsilon, \epsilon)$ \COMMENT{Random start}
\FOR{$t = 0$ to $T-1$}
    \STATE Compute loss: $L = J(\theta, x^{adv}_t, y)$
    \STATE Compute gradient: $g = \nabla_x L$
    \STATE Update: $x^{adv}_{t+1} = x^{adv}_t + \alpha \cdot \text{sign}(g)$
    \STATE Project: $x^{adv}_{t+1} = \Pi_{x,\epsilon}(x^{adv}_{t+1})$ \COMMENT{Project to $\epsilon$-ball}
    \STATE Clip: $x^{adv}_{t+1} = \text{clip}(x^{adv}_{t+1}, 0, 1)$
\ENDFOR
\RETURN $x^{adv}_T$
\end{algorithmic}
\end{algorithm}

The implementation begins with random initialization within the $\epsilon$-ball, which is important for finding diverse adversarial examples and avoiding local optima. At each iteration, the gradient of the cross-entropy loss with respect to the current adversarial image is computed, and a step of size $\alpha$ is taken in the sign direction of the gradient. The projection step clamps the perturbation $x^{adv} - x$ to the range $[-\epsilon, \epsilon]$ element-wise, ensuring the $L_\infty$ constraint is satisfied. The result is further clamped to $[0, 1]$ for valid pixel values. Gradients are detached after each iteration to prevent computational graph accumulation, which is important for memory efficiency during multi-step attacks.

\subsection{DeepFool Implementation}

DeepFool~\cite{moosavi2016deepfool} was implemented as an iterative algorithm that finds the minimal perturbation needed to change the model's prediction. Unlike FGSM and PGD, which operate with a fixed perturbation budget, DeepFool computes the closest decision boundary and pushes the input just past it.

The algorithm operates on individual images (not batches) because each image has a different closest decision boundary. At each iteration, the algorithm computes the gradient of every class logit with respect to the input. For each non-original class $k$, it calculates the distance to the corresponding decision boundary using the linearized approximation:

\begin{equation}
    d_k = \frac{|f_k(x) - f_{\hat{k}}(x)|}{\|w_k\|_2}
\end{equation}

where $\hat{k}$ is the original predicted class and $w_k = \nabla_x f_k(x) - \nabla_x f_{\hat{k}}(x)$ is the difference in gradients. The algorithm selects the class with the smallest distance and computes the minimum perturbation to cross that boundary. This process repeats until the predicted class changes or the maximum iteration count is reached. An overshoot parameter of 0.02 is applied to the total perturbation to ensure the input crosses the boundary reliably.

The per-image processing makes DeepFool significantly slower than batched attacks, but it provides the valuable property of finding near-minimal perturbations, giving a measure of each sample's \textit{robustness radius}---the distance to the nearest decision boundary.

\subsection{Adversarial Training Implementation}

Adversarial training was implemented following the PGD-based approach of Madry et al.~\cite{madry2018towards}. The key modification to the standard training loop is that at each training step, PGD adversarial examples are generated from the current mini-batch and the model is trained on these adversarial examples rather than the clean inputs.

A subtle but important implementation detail involves batch normalization layers. During adversarial example generation, the model is temporarily switched to evaluation mode to ensure consistent batch normalization statistics. After generating the adversarial batch, the model is switched back to training mode for the parameter update step. This prevents the attack generation process from corrupting the running statistics of batch normalization layers.

The adversarial training loop uses SGD with momentum (0.9) and weight decay ($5 \times 10^{-4}$), with a cosine annealing learning rate schedule starting from 0.1. Model checkpoints are saved based on PGD robustness accuracy rather than clean accuracy, since the goal of adversarial training is to maximize robust performance. All training metrics (loss, clean accuracy, robust accuracy, learning rate) are logged to MLflow at each epoch for experiment tracking and comparison.

\section{Development Environment}

The development and training were conducted using the following environment:

\begin{itemize}
    \item \textbf{Operating System:} macOS (Apple Silicon)
    \item \textbf{Python Version:} 3.13
    \item \textbf{PyTorch Version:} 2.x with MPS (Metal Performance Shaders) backend
    \item \textbf{Hardware:} Apple M4 Pro, 16GB unified memory
    \item \textbf{Experiment Tracking:} MLflow
    \item \textbf{Deployment:} Docker, HuggingFace Spaces
\end{itemize}

Training was performed using Apple's Metal Performance Shaders (MPS) backend for GPU acceleration on Apple Silicon. Standard training for 50 epochs completed in approximately 55 minutes, while adversarial training required approximately 4.5 hours due to the computational overhead of generating PGD attacks at every training step (approximately 7$\times$ slower per epoch).

\section{Challenges and Solutions}

Several implementation challenges were encountered during development:

\textbf{Numerical stability in DeepFool:} Computing gradients for all classes simultaneously required careful management of the computational graph using \texttt{retain\_graph=True} during backward passes. A small epsilon ($10^{-8}$) was added to norm computations to prevent division by zero when gradient differences were near-zero.

\textbf{Memory management during PGD:} The multi-step PGD attack accumulates a computational graph across iterations if gradients are not properly detached. The solution was to call \texttt{.detach()} on the adversarial images after each projection step, preventing graph accumulation while maintaining gradient flow within each iteration.

\textbf{Batch normalization during adversarial training:} Generating adversarial examples with the model in training mode caused the batch normalization running statistics to be corrupted by adversarial inputs. The solution was to switch to evaluation mode during attack generation and back to training mode for the actual parameter update.

\textbf{Apple Silicon compatibility:} Some PyTorch operations had limited MPS support at the time of development, requiring occasional fallback to CPU for specific operations and careful testing of numerical equivalence between MPS and CPU results.

\section{Code Organization}

The codebase follows a modular structure organized by functionality:

\begin{verbatim}
adversarial-robustness/
+-- src/
|   +-- models/
|   |   +-- resnet.py          # ResNet-18 for CIFAR-10
|   +-- attacks/
|   |   +-- fgsm.py            # FGSM attack
|   |   +-- pgd.py             # PGD attack
|   |   +-- deepfool.py        # DeepFool attack
|   +-- training/
|   |   +-- standard.py        # Standard training loop
|   |   +-- adversarial.py     # PGD adversarial training
|   +-- utils/
|   |   +-- data.py            # Data loading and transforms
|   |   +-- metrics.py         # Evaluation metrics
|   +-- api/
|       +-- main.py            # FastAPI backend
+-- frontend/
|   +-- app.py                 # Streamlit demo interface
+-- scripts/
|   +-- train.py               # Training entry point
|   +-- evaluate.py            # Evaluation entry point
+-- configs/
|   +-- config.yaml            # Hyperparameters
+-- checkpoints/               # Saved model weights
+-- tests/
|   +-- test_attacks.py        # Unit tests
+-- Dockerfile                 # Container for deployment
\end{verbatim}

The entry points (\texttt{scripts/train.py} and \texttt{scripts/evaluate.py}) accept command-line arguments to select training mode (standard, adversarial, or both) and evaluation configurations. Configuration parameters are loaded from a YAML file, ensuring reproducibility across experiments.

\section{Deployment}

To demonstrate the practical applicability of the research, an interactive web application was developed and deployed. The frontend uses Streamlit to provide a user interface where visitors can select sample CIFAR-10 images or upload their own, choose an attack method and perturbation budget, and observe the effects on both the standard and adversarially trained models in real time.

The application displays the original image, the adversarial perturbation (amplified for visibility), and the adversarial image side by side, along with confidence scores for both models. This visualization makes the abstract concept of adversarial vulnerability tangible and accessible. The application is containerized with Docker and deployed on HuggingFace Spaces for public access.

\section{Summary}

This chapter has described the complete implementation of the adversarial attack and defense framework. The modular design separates model architectures, attack algorithms, training procedures, and evaluation into independent components with consistent interfaces. Key implementation decisions---such as keeping inputs in $[0, 1]$ pixel space, managing batch normalization during adversarial training, and using MLflow for experiment tracking---were driven by both the requirements of adversarial robustness research and software engineering best practices. The following chapter presents the experimental results obtained using this implementation.
\chapter{RESULTS AND DISCUSSION}
\label{ch:results}

\section{Introduction}

In this chapter we present the results of our experiments. We evaluate all three attacks on CIFAR-10, report the effect of adversarial training, and compare the attacks under identical conditions. All experiments used the framework described in Chapter~\ref{ch:implementation}, with results tracked via MLflow.

\section{Experimental Setup}

\subsection{Training Configuration}

We trained two ResNet-18 models on CIFAR-10: one with standard training and one with PGD adversarial training. Both models used identical hyperparameters except for the adversarial training component. Table~\ref{tab:training_config} lists the training configuration.

\begin{table}[htbp]
    \centering
    \caption{Training hyperparameters for standard and adversarial training}
    \label{tab:training_config}
    \begin{tabular}{lcc}
        \toprule
        \textbf{Parameter} & \textbf{Standard Training} & \textbf{Adversarial Training} \\
        \midrule
        Batch size & 128 & 128 \\
        Optimizer & SGD & SGD \\
        Learning rate & 0.1 & 0.1 \\
        Momentum & 0.9 & 0.9 \\
        Weight decay & $5 \times 10^{-4}$ & $5 \times 10^{-4}$ \\
        LR schedule & Cosine annealing & Cosine annealing \\
        Epochs & 50 & 50 \\
        Data augmentation & RandomCrop, HFlip & RandomCrop, HFlip \\
        \bottomrule
    \end{tabular}
\end{table}

Standard training took approximately 55 minutes. Adversarial training took approximately 4.5 hours, roughly $7\times$ slower due to the PGD computation at each training step.

\subsection{Attack Parameters}

Table~\ref{tab:attack_params} lists the attack configurations used during evaluation.

\begin{table}[htbp]
    \centering
    \caption{Attack parameters used during evaluation}
    \label{tab:attack_params}
    \begin{tabular}{lp{8cm}}
        \toprule
        \textbf{Attack} & \textbf{Parameters} \\
        \midrule
        FGSM & $\epsilon \in \{2/255, 4/255, 8/255, 16/255\}$ \\
        PGD & $\epsilon \in \{2/255, 4/255, 8/255, 16/255\}$, step size $\alpha = \epsilon/4$, iterations $T = 20$, random start \\
        DeepFool & Max iterations = 50, overshoot = 0.02, 10 classes \\
        \bottomrule
    \end{tabular}
\end{table}

For the inner PGD attack during adversarial training, we used $\epsilon = 8/255$, $\alpha = 2/255$, 7 iterations, and random start, following Madry et al.~\cite{madry2018towards}.

\subsection{Evaluation Protocol}

All evaluations were performed on the full CIFAR-10 test set (10,000 images). PGD used one random restart, which is standard for evaluating adversarially trained models. Attack success rates were computed only over images that the model classified correctly on clean data.

We also built an interactive web demo on HuggingFace Spaces (Figure~\ref{fig:demo_interface}) that allows comparison of both models under any attack and $\epsilon$ value.

\begin{figure}[htbp]
    \centering
    \includegraphics[width=0.95\textwidth]{figures/stats.png}
    \caption{Interactive demo interface deployed on HuggingFace Spaces, showing model accuracy metrics and attack configuration controls for real-time adversarial robustness evaluation.}
    \label{fig:demo_interface}
\end{figure}

\section{Baseline Results}

\subsection{Model Performance on Clean Data}

Table~\ref{tab:clean_accuracy} reports the clean test accuracy of both models.

\begin{table}[htbp]
    \centering
    \caption{Clean accuracy of trained models on CIFAR-10}
    \label{tab:clean_accuracy}
    \begin{tabular}{lc}
        \toprule
        \textbf{Model} & \textbf{Clean Accuracy (\%)} \\
        \midrule
        Standard ResNet-18 & 94.1 \\
        Adversarially Trained ResNet-18 & 83.9 \\
        \bottomrule
    \end{tabular}
\end{table}

The standard model achieves 94.1\%, which is consistent with published results for ResNet-18 on CIFAR-10 without input normalization. The adversarially trained model achieves 83.9\%, a reduction of 10.2 percentage points. This trade-off is consistent with prior work~\cite{madry2018towards} and is discussed further in Section~\ref{sec:tradeoffs}.

\section{Attack Evaluation Results}

\subsection{FGSM Attack Results}

Table~\ref{tab:fgsm_results} reports robust accuracy under FGSM at varying $\epsilon$.

\begin{table}[htbp]
    \centering
    \caption{Robust accuracy (\%) under FGSM attack on CIFAR-10}
    \label{tab:fgsm_results}
    \begin{tabular}{lcccc}
        \toprule
        \textbf{Model} & $\epsilon=2/255$ & $\epsilon=4/255$ & $\epsilon=8/255$ & $\epsilon=16/255$ \\
        \midrule
        Standard & 68.2 & 42.5 & 18.3 & 5.1 \\
        Adv-Trained & 76.8 & 68.4 & 56.0 & 38.2 \\
        \bottomrule
    \end{tabular}
\end{table}

We observe that the standard model drops from 94.1\% to 18.3\% at $\epsilon = 8/255$. The adversarially trained model retains 56.0\% accuracy, a gain of 37.7 percentage points. Figure~\ref{fig:attack_fgsm} shows visual examples.

\subsection{PGD Attack Results}

Table~\ref{tab:pgd_results} reports results under PGD-20. This is the stronger evaluation.

\begin{table}[htbp]
    \centering
    \caption{Robust accuracy (\%) under PGD-20 attack on CIFAR-10}
    \label{tab:pgd_results}
    \begin{tabular}{lcccc}
        \toprule
        \textbf{Model} & $\epsilon=2/255$ & $\epsilon=4/255$ & $\epsilon=8/255$ & $\epsilon=16/255$ \\
        \midrule
        Standard & 41.3 & 12.8 & 1.2 & 0.1 \\
        Adv-Trained & 72.1 & 58.3 & 45.1 & 22.6 \\
        \bottomrule
    \end{tabular}
\end{table}

We find that the standard model retains only 1.2\% accuracy at $\epsilon = 8/255$ under PGD, compared to 18.3\% under FGSM. The single-step attack gives a misleading picture of the model's actual vulnerability. The adversarially trained model retains 45.1\% at the same $\epsilon$. At $\epsilon = 16/255$, the standard model is reduced to 0.1\% (effectively random), while the adversarially trained model still achieves 22.6\%. See Figure~\ref{fig:attack_pgd}.

\subsection{DeepFool Attack Results}

DeepFool finds the minimum perturbation per image rather than operating within a fixed budget. Table~\ref{tab:deepfool_results} reports the results.

\begin{table}[htbp]
    \centering
    \caption{DeepFool attack results on CIFAR-10}
    \label{tab:deepfool_results}
    \begin{tabular}{lccc}
        \toprule
        \textbf{Model} & \textbf{Success Rate (\%)} & \textbf{Avg. $L_2$ Perturbation} & \textbf{Avg. $L_\infty$ Perturbation} \\
        \midrule
        Standard & 91.1 & 0.248 & 0.021 \\
        Adv-Trained & 62.0 & 0.892 & 0.074 \\
        \bottomrule
    \end{tabular}
\end{table}

DeepFool fools 91.1\% of correctly classified images on the standard model with an average $L_2$ perturbation of only 0.248. This indicates that the decision boundaries of the standard model are very close to the data points. For the adversarially trained model, DeepFool requires approximately $3.6\times$ larger perturbations and achieves a 62.0\% success rate. The decision boundaries have moved further from the data. See Figure~\ref{fig:attack_deepfool}.

\subsection{Visual Examples}

Figure~\ref{fig:adversarial_examples} shows a CIFAR-10 image attacked with FGSM and PGD at $\epsilon = 8/255$. The perturbation is imperceptible, yet the standard model changes its prediction with high confidence.

\begin{figure}[htbp]
    \centering
    \begin{subfigure}[b]{0.95\textwidth}
        \centering
        \includegraphics[width=\textwidth]{figures/FGSM.png}
        \caption{FGSM attack ($\epsilon = 8/255$): the standard model is fooled with near-100\% confidence on the wrong class, while the robust model correctly identifies the cat with approximately 52\% confidence.}
        \label{fig:attack_fgsm}
    \end{subfigure}

    \vspace{0.5cm}

    \begin{subfigure}[b]{0.95\textwidth}
        \centering
        \includegraphics[width=\textwidth]{figures/PGD.png}
        \caption{PGD attack ($\epsilon = 8/255$, 20 steps): the stronger iterative attack also fools the standard model with near-100\% confidence, while the robust model maintains the correct prediction.}
        \label{fig:attack_pgd}
    \end{subfigure}

    \caption{Visual comparison of FGSM and PGD adversarial attacks on a CIFAR-10 cat image. Each panel shows the original image, the adversarial perturbation (amplified for visibility), the adversarial image, and the predictions of both the standard model (fooled) and the robust model (correct). The perturbations are imperceptible to human observers yet cause confident misclassification in the standard model.}
    \label{fig:adversarial_examples}
\end{figure}

\begin{figure}[htbp]
    \centering
    \includegraphics[width=0.95\textwidth]{figures/DeepFool.png}
    \caption{DeepFool attack on the same CIFAR-10 cat image: finds the minimal perturbation to cross the decision boundary. The standard model is fooled, while the robust model resists the minimal perturbation. Unlike FGSM and PGD (Figure~\ref{fig:adversarial_examples}), DeepFool does not use a fixed $\epsilon$ budget but instead computes the smallest perturbation needed to change the prediction.}
    \label{fig:attack_deepfool}
\end{figure}

We also note a pattern in the confidence scores. The standard model, when fooled, outputs near-100\% confidence on the wrong class. It does not indicate any uncertainty. The adversarially trained model, by contrast, tends to produce confidence scores in the 50--60\% range even on correct predictions. This suggests that adversarial training produces better-calibrated confidence outputs --- the model assigns lower confidence when the input is ambiguous or perturbed.

\section{Adversarial Training Results}

\subsection{Training Dynamics}

For standard training, the loss decreased from approximately 1.55 to 0.034 over 50 epochs. Training accuracy reached 98\% and test accuracy settled at 94.1\%. We observed no overfitting.

Adversarial training exhibited different dynamics. The training loss remained higher throughout, which is expected since the model is trained on adversarial examples that are harder to classify. Clean test accuracy settled at 83.9\%. PGD robust accuracy, which we used for checkpoint selection, peaked at 50.6\%.

\subsection{PGD Adversarial Training on CIFAR-10}

Table~\ref{tab:adv_training_comparison} compares the two models at $\epsilon = 8/255$.

\begin{table}[htbp]
    \centering
    \caption{Comprehensive comparison at $\epsilon = 8/255$ on CIFAR-10}
    \label{tab:adv_training_comparison}
    \begin{tabular}{lccccc}
        \toprule
        \textbf{Model} & \textbf{Clean} & \textbf{FGSM} & \textbf{PGD-20} & \textbf{DeepFool} & \textbf{Avg. Robust} \\
        \midrule
        Standard & 94.1 & 18.3 & 1.2 & 8.9 & 9.5 \\
        Adv-Trained & 83.9 & 56.0 & 45.1 & 38.0 & 46.4 \\
        \midrule
        Improvement & $-10.2$ & $+37.7$ & $+43.9$ & $+29.1$ & $+36.9$ \\
        \bottomrule
    \end{tabular}
\end{table}

The adversarially trained model loses 10.2 percentage points of clean accuracy but gains an average of 36.9 points across the three robustness metrics. The largest gain is on PGD (+43.9 pp), which is expected since PGD is the attack used during training. The gains on FGSM and DeepFool indicate that robustness transfers across attack methods.

\subsection{Accuracy Trade-offs}
\label{sec:tradeoffs}

The 10.2 percentage point reduction in clean accuracy is consistent with known results on the accuracy--robustness trade-off. We remark that the practical question is whether a model with 94.1\% clean accuracy and 1.2\% robust accuracy is preferable to one with 83.9\% clean accuracy and 45.1\% robust accuracy. For safety-critical applications, the latter is clearly more appropriate.

The accuracy reduction arises because adversarial training encourages the model to learn features that are stable under worst-case perturbations, and these features are not necessarily the same ones that maximize clean accuracy. There is a genuine tension between these two objectives, and the 10.2 percentage point cost reflects this tension on CIFAR-10 with ResNet-18.

\subsection{Computational Cost}

Adversarial training is approximately $7\times$ slower per epoch than standard training: 55 minutes versus 4.5 hours for 50 epochs. This is because 7-step PGD at each training batch requires 7 additional forward and backward passes. On our M4 Pro hardware this was manageable for CIFAR-10, but it would become a bottleneck for larger datasets and models.

\section{Attack Comparison}

\subsection{Success Rate and Strength Comparison}

Table~\ref{tab:attack_comparison} compares all three attacks on the standard model at $\epsilon = 8/255$.

\begin{table}[htbp]
    \centering
    \caption{Comparison of attack methods on CIFAR-10 standard model ($\epsilon = 8/255$)}
    \label{tab:attack_comparison}
    \begin{tabular}{lcccc}
        \toprule
        \textbf{Attack} & \textbf{Robust Acc. (\%)} & \textbf{Avg. $L_2$} & \textbf{Avg. $L_\infty$} & \textbf{Time per batch (ms)} \\
        \midrule
        FGSM & 18.3 & 1.351 & 0.031 & 12 \\
        PGD-20 & 1.2 & 1.351 & 0.031 & 185 \\
        DeepFool & 8.9 & 0.248 & 0.021 & 2,450 \\
        \bottomrule
    \end{tabular}
\end{table}

PGD is the strongest attack, reducing accuracy to 1.2\% compared to 18.3\% for FGSM and 8.9\% for DeepFool. We note that FGSM and PGD operate under the same $L_\infty$ budget, but PGD finds stronger perturbations within that budget through iteration. DeepFool uses substantially less perturbation ($L_2 = 0.248$ versus 1.351) because it seeks the minimum perturbation rather than maximum damage within a fixed budget.

In terms of computational cost, FGSM requires 12 ms per batch, PGD requires 185 ms (20 gradient steps), and DeepFool requires 2,450 ms due to per-image processing with per-class gradients. PGD offers the best trade-off between attack strength and computational cost for evaluation purposes. DeepFool provides complementary information (the robustness radius) but is too slow for routine evaluation.

\subsection{Perturbation Analysis}

FGSM and PGD both saturate the $\epsilon$ budget, perturbing every pixel by exactly $\epsilon$ in the gradient direction. DeepFool, by contrast, uses $5.4\times$ less $L_2$ perturbation on average while still fooling 91.1\% of images. This indicates that the decision boundaries of the standard model lie very close to the data --- the average distance from a data point to the nearest wrong-class boundary is 0.248 in $L_2$.

After adversarial training, DeepFool requires an average of 0.892 $L_2$ perturbation, approximately $3.6\times$ larger. This confirms that adversarial training moves the decision boundaries further from the data.

\section{Discussion}

\subsection{Key Findings}

We summarize our main findings:

First, the standard model is highly vulnerable. It achieves 94.1\% clean accuracy but only 1.2\% under PGD at $\epsilon = 8/255$, with imperceptible perturbations.

Second, single-step attacks underestimate the true vulnerability. FGSM gives 18.3\% robust accuracy where PGD gives 1.2\%. Evaluating with FGSM alone would give a misleading picture of the model's actual robustness.

Third, adversarial training provides real robustness. PGD robust accuracy increases from 1.2\% to 45.1\%, at a cost of 10.2 percentage points of clean accuracy. Our numbers are consistent with Madry et al.~\cite{madry2018towards}, which provides confidence that the implementation is correct.

Fourth, adversarial training improves confidence calibration. The standard model assigns near-100\% confidence to incorrect predictions under attack. The adversarially trained model produces more distributed confidence scores, which is useful for detecting unreliable predictions.

\subsection{Comparison with Literature}

Our results are consistent with published numbers. Madry et al.~\cite{madry2018towards} reported 45--47\% PGD robust accuracy at $\epsilon = 8/255$ on CIFAR-10 with adversarial training; we obtain 45.1\%. Our clean accuracy of 83.9\% falls within the 82--87\% range reported in the literature. The DeepFool robustness radii are consistent with Moosavi-Dezfooli et al.~\cite{moosavidezfooli2016deepfool}, confirming that standard networks have small margins around the data.

\subsection{Limitations and Observations}

We note several limitations. Our experiments are restricted to CIFAR-10 ($32 \times 32$ images), and the same $\epsilon$ budget has different perceptual meaning at higher resolutions. We tested only ResNet-18; other architectures may exhibit different robustness properties. We used one PGD random restart; more restarts would provide tighter bounds on vulnerability. We evaluated only $L_\infty$-bounded attacks; $L_2$-bounded attacks may yield different conclusions.

\section{Summary}

We find that standard models are highly vulnerable to adversarial perturbations, and that PGD reveals this vulnerability more accurately than FGSM. Adversarial training provides genuine robustness at the cost of clean accuracy. Our results are consistent with the literature, and the comparison across three attack methods provides a unified view of the accuracy--robustness trade-off.


\chapter{CONCLUSION AND FUTURE WORK}
\label{ch:conclusion}

\section{Conclusion}

This thesis set out to answer a simple question: how vulnerable are standard deep learning models to adversarial attacks, and can adversarial training do anything about it? To find out, I implemented three white-box attacks --- FGSM, PGD, and DeepFool --- tested them against a ResNet-18 trained on CIFAR-10, and then evaluated PGD-based adversarial training as a defense.

The answer, in short: standard models are extremely fragile, and adversarial training helps substantially but not for free. My ResNet-18 achieves 94.1\% accuracy on clean data. A 20-step PGD attack at $\epsilon = 8/255$ --- a perturbation invisible to the human eye --- drops that to 1.2\%. FGSM, being a weaker attack, only gets the model down to 18.3\%, which would give a misleadingly optimistic picture of the model's security if used as the sole evaluation. The gap between these two numbers (18.3\% vs 1.2\%) illustrates why multi-step attacks like PGD are necessary for honest robustness evaluation.

Adversarial training improves the picture considerably. The PGD-trained model achieves 45.1\% robust accuracy under PGD attack, up from near-zero, while giving up 10.2 percentage points of clean accuracy. For safety-critical applications, this seems like a reasonable trade-off.

One additional observation: the adversarially trained model produces noticeably better-calibrated confidence scores. The standard model outputs near-100\% confidence on its wrong adversarial predictions. The robust model spreads its probability mass more evenly, which means you can actually tell when it is uncertain. In practice, this calibration property could be quite valuable.

\section{Research Contributions}

Concretely, this thesis contributes the following:

\begin{enumerate}
    \item \textbf{Attack comparison:} A side-by-side evaluation of FGSM, PGD, and DeepFool on the same model with the same protocol. PGD was the strongest attack (1.2\% surviving accuracy at $\epsilon = 8/255$). FGSM was substantially weaker (18.3\%). DeepFool achieved a 91.1\% fooling rate with perturbations $5.4\times$ smaller in $L_2$ norm than FGSM/PGD.

    \item \textbf{Adversarial training evaluation:} The adversarially trained model achieved 45.1\% robust accuracy under PGD and 83.9\% clean accuracy --- numbers that match the benchmarks reported by Madry et al., confirming the correctness of my implementation.

    \item \textbf{Open-source framework:} The complete codebase is publicly available on GitHub: three attack implementations, two training pipelines, MLflow integration, and an interactive Streamlit demo on HuggingFace Spaces. The modular design means adding a new attack is a matter of writing one Python file.

    \item \textbf{Calibration observation:} The robust model not only gets adversarial examples right more often --- it also produces more honest confidence scores. The standard model says ``99\% cat'' when it is wrong; the robust model says ``55\% cat'' when it is right.
\end{enumerate}

\section{Limitations}

This work has clear limitations:

\begin{enumerate}
    \item \textbf{Only CIFAR-10.} Images are $32 \times 32$ with 10 classes. An $\epsilon = 8/255$ perturbation on a 32-pixel image is perceptually different from the same $\epsilon$ on a 224-pixel image. I cannot claim these results generalize to ImageNet-scale data without running those experiments.

    \item \textbf{Only ResNet-18.} Other architectures --- Vision Transformers, for instance --- may exhibit different vulnerability profiles. I did not test any alternatives.

    \item \textbf{Only white-box $L_\infty$.} No black-box attacks, no transfer attacks, no $L_2$ or $L_0$ threat models. I also did not implement Carlini-Wagner due to its computational requirements on my hardware.

    \item \textbf{Only one defense.} PGD adversarial training is one option among several. Certified defenses, randomized smoothing, and input preprocessing were not evaluated.

    \item \textbf{Hardware constraints.} All experiments ran on an Apple M4 Pro laptop. Adversarial training took 4.5 hours for 50 epochs, leaving little room for hyperparameter exploration.
\end{enumerate}

\section{Future Work}

Several extensions would strengthen and broaden these results.

\subsection{Short-term}

\begin{itemize}
    \item \textbf{Add Carlini-Wagner attack.} C\&W is widely considered the gold standard for evaluating defenses. Testing the adversarially trained model against it would provide a more complete robustness picture.

    \item \textbf{MNIST experiments.} Running the same evaluation pipeline on MNIST would enable comparison with a large body of existing results and show how dataset complexity affects vulnerability.

    \item \textbf{Transferability tests.} Do adversarial examples crafted for the standard model transfer to the robust model? This is relevant to black-box threat scenarios and I did not test it.

    \item \textbf{Multiple PGD restarts.} I used a single random restart for evaluation. Multiple restarts would give tighter robustness bounds.
\end{itemize}

\subsection{Long-term}

\begin{itemize}
    \item \textbf{ImageNet scale.} The key question is whether these findings hold on larger images, more classes, and bigger models.

    \item \textbf{Certified defenses.} Randomized smoothing provides formal robustness guarantees. Comparing it with the empirical approach taken here would be informative.

    \item \textbf{Domain-specific evaluation.} Medical imaging and autonomous driving are the application areas where adversarial robustness matters most. Experiments on X-ray images or traffic sign datasets would be more directly relevant than CIFAR-10.

    \item \textbf{Architecture comparison.} The question of whether Vision Transformers are more or less vulnerable than CNNs does not have a settled answer. My framework could be extended to investigate this.

    \item \textbf{Faster adversarial training.} 4.5 hours for 50 epochs on CIFAR-10 is already slow. Methods like Adversarial Training for Free~\cite{shafahi2019adversarial} could reduce the overhead significantly.
\end{itemize}

\section{Final Remarks}

Adversarial vulnerability remains one of the central unsolved problems in deploying deep learning to the real world. The experiments in this thesis make the severity of the problem concrete: a 94\% accurate model can be reduced to near-zero by perturbations that are invisible to the human eye. Adversarial training offers a practical defense, but it comes with a cost in both clean accuracy and computational budget.

As deep learning continues its expansion into autonomous vehicles, medical diagnostics, security systems, and other high-stakes domains, addressing adversarial robustness will only grow more urgent. I hope the implementations, experimental results, and interactive demo from this work can be useful --- both as a practical reference and as an illustration of why this line of research matters.


% Bibliography
\bibliography{references}

% Appendices (if any)
\appendix
\chapter{SOURCE CODE}
\label{ch:appendix_code}

This appendix provides the core Python implementations for the adversarial attack algorithms, training procedures, and model architecture discussed in Chapter~\ref{ch:implementation}. The complete codebase is available at \url{https://github.com/Brlr2003/adversarial-robustness}.

\section{ResNet-18 Model Architecture}

\begin{lstlisting}[language=Python, caption={ResNet-18 Architecture for CIFAR-10}, label={lst:resnet}]
import torch
import torch.nn as nn
import torch.nn.functional as F


class BasicBlock(nn.Module):
    """Basic residual block with two 3x3 convolutions."""
    expansion = 1

    def __init__(self, in_channels, out_channels, stride=1):
        super().__init__()
        self.conv1 = nn.Conv2d(in_channels, out_channels,
            kernel_size=3, stride=stride, padding=1, bias=False)
        self.bn1 = nn.BatchNorm2d(out_channels)
        self.conv2 = nn.Conv2d(out_channels, out_channels,
            kernel_size=3, stride=1, padding=1, bias=False)
        self.bn2 = nn.BatchNorm2d(out_channels)

        self.shortcut = nn.Sequential()
        if stride != 1 or in_channels != out_channels:
            self.shortcut = nn.Sequential(
                nn.Conv2d(in_channels, out_channels,
                    kernel_size=1, stride=stride, bias=False),
                nn.BatchNorm2d(out_channels),
            )

    def forward(self, x):
        out = F.relu(self.bn1(self.conv1(x)))
        out = self.bn2(self.conv2(out))
        out += self.shortcut(x)
        out = F.relu(out)
        return out


class ResNet(nn.Module):
    """ResNet adapted for CIFAR-10 (32x32 inputs)."""
    def __init__(self, block, num_blocks, num_classes=10):
        super().__init__()
        self.in_channels = 64
        # CIFAR-10: 3x3 conv instead of 7x7, no maxpool
        self.conv1 = nn.Conv2d(3, 64, kernel_size=3,
            stride=1, padding=1, bias=False)
        self.bn1 = nn.BatchNorm2d(64)
        self.layer1 = self._make_layer(block, 64,
            num_blocks[0], stride=1)
        self.layer2 = self._make_layer(block, 128,
            num_blocks[1], stride=2)
        self.layer3 = self._make_layer(block, 256,
            num_blocks[2], stride=2)
        self.layer4 = self._make_layer(block, 512,
            num_blocks[3], stride=2)
        self.avg_pool = nn.AdaptiveAvgPool2d((1, 1))
        self.fc = nn.Linear(512, num_classes)

    def _make_layer(self, block, out_channels,
                    num_blocks, stride):
        strides = [stride] + [1] * (num_blocks - 1)
        layers = []
        for s in strides:
            layers.append(block(self.in_channels,
                out_channels, s))
            self.in_channels = out_channels
        return nn.Sequential(*layers)

    def forward(self, x):
        out = F.relu(self.bn1(self.conv1(x)))
        out = self.layer1(out)
        out = self.layer2(out)
        out = self.layer3(out)
        out = self.layer4(out)
        out = self.avg_pool(out)
        out = torch.flatten(out, 1)
        out = self.fc(out)
        return out


def resnet18_cifar10(num_classes=10):
    return ResNet(BasicBlock, [2, 2, 2, 2], num_classes)
\end{lstlisting}

\section{FGSM Attack Implementation}

\begin{lstlisting}[language=Python, caption={FGSM Attack Implementation}, label={lst:fgsm}]
import torch
import torch.nn as nn


def fgsm_attack(model, images, labels, epsilon,
                device="cpu"):
    """
    Fast Gradient Sign Method (FGSM) attack.

    Args:
        model: Target neural network
        images: Clean input images (B, C, H, W) in [0, 1]
        labels: True labels (B,)
        epsilon: Maximum perturbation (L_inf bound)
        device: Device to run on

    Returns:
        Adversarial images (B, C, H, W) in [0, 1]
    """
    images = images.clone().detach().to(device)
    labels = labels.clone().detach().to(device)
    images.requires_grad = True

    # Forward pass
    outputs = model(images)
    loss = nn.CrossEntropyLoss()(outputs, labels)

    # Backward pass
    model.zero_grad()
    loss.backward()

    # Generate adversarial examples
    adv_images = images + epsilon * images.grad.sign()
    adv_images = adv_images.clamp(0, 1).detach()

    return adv_images
\end{lstlisting}

\section{PGD Attack Implementation}

\begin{lstlisting}[language=Python, caption={PGD Attack Implementation}, label={lst:pgd}]
import torch
import torch.nn as nn


def pgd_attack(model, images, labels, epsilon, alpha,
               steps, random_start=True, device="cpu"):
    """
    Projected Gradient Descent (PGD) attack.

    Args:
        model: Target neural network
        images: Clean input images (B, C, H, W) in [0, 1]
        labels: True labels (B,)
        epsilon: Maximum perturbation (L_inf bound)
        alpha: Step size per iteration
        steps: Number of PGD iterations
        random_start: Start from random point in ball
        device: Device to run on

    Returns:
        Adversarial images (B, C, H, W) in [0, 1]
    """
    images = images.clone().detach().to(device)
    labels = labels.clone().detach().to(device)
    adv_images = images.clone().detach()

    if random_start:
        adv_images = adv_images + torch.empty_like(
            adv_images).uniform_(-epsilon, epsilon)
        adv_images = adv_images.clamp(0, 1).detach()

    loss_fn = nn.CrossEntropyLoss()

    for _ in range(steps):
        adv_images.requires_grad = True
        outputs = model(adv_images)
        loss = loss_fn(outputs, labels)

        model.zero_grad()
        loss.backward()

        # Step in gradient direction
        grad_sign = adv_images.grad.sign()
        adv_images = adv_images.detach() + alpha * grad_sign

        # Project back into epsilon-ball
        perturbation = torch.clamp(
            adv_images - images, min=-epsilon, max=epsilon)
        adv_images = (images + perturbation).clamp(
            0, 1).detach()

    return adv_images
\end{lstlisting}

\section{DeepFool Attack Implementation}

\begin{lstlisting}[language=Python, caption={DeepFool Attack Implementation}, label={lst:deepfool}]
import torch
import torch.nn as nn


def deepfool_attack(model, images, max_iterations=50,
                    overshoot=0.02, num_classes=10,
                    device="cpu"):
    """
    DeepFool attack: finds minimal perturbation to
    cross the nearest decision boundary.

    Args:
        model: Target neural network
        images: Clean input images (B, C, H, W)
        max_iterations: Maximum iterations per image
        overshoot: Overshoot parameter
        num_classes: Number of classes
        device: Device to run on

    Returns:
        (adversarial_images, perturbation_norms)
    """
    model.eval()
    images = images.clone().detach().to(device)
    batch_size = images.shape[0]
    adv_images = images.clone()
    perturbation_norms = torch.zeros(batch_size,
                                     device=device)

    for idx in range(batch_size):
        x = images[idx:idx+1].clone().requires_grad_(True)
        output = model(x)
        orig_class = output.argmax(dim=1).item()
        total_pert = torch.zeros_like(x)

        for _ in range(max_iterations):
            x_pert = (images[idx:idx+1] + total_pert
                     ).requires_grad_(True)
            output = model(x_pert)
            if output.argmax(dim=1).item() != orig_class:
                break

            # Compute gradients for all classes
            gradients = []
            for k in range(num_classes):
                if x_pert.grad is not None:
                    x_pert.grad.zero_()
                output[0, k].backward(retain_graph=True)
                gradients.append(x_pert.grad.clone())

            # Find closest decision boundary
            f_orig = output[0, orig_class]
            min_dist = float("inf")
            best_pert = None

            for k in range(num_classes):
                if k == orig_class:
                    continue
                w_k = gradients[k] - gradients[orig_class]
                f_diff = (output[0, k] - f_orig
                         ).abs().item()
                w_norm = w_k.flatten().norm().item()
                if w_norm == 0:
                    continue
                dist = f_diff / w_norm
                if dist < min_dist:
                    min_dist = dist
                    best_pert = (f_diff / (w_norm**2
                        + 1e-8)) * w_k

            if best_pert is None:
                break
            total_pert += best_pert.detach()

        total_pert = (1 + overshoot) * total_pert
        adv_images[idx] = (images[idx]
            + total_pert.squeeze(0)).clamp(0, 1)
        perturbation_norms[idx] = (
            total_pert.flatten().norm())

    return adv_images.detach(), perturbation_norms.detach()
\end{lstlisting}

\section{Adversarial Training Implementation}

\begin{lstlisting}[language=Python, caption={PGD Adversarial Training Loop}, label={lst:adv_training}]
import torch
import torch.nn as nn
import torch.optim as optim
from torch.optim.lr_scheduler import CosineAnnealingLR
from src.attacks.pgd import pgd_attack


def train_adversarial(model, train_loader, test_loader,
                      config, adv_config, device,
                      checkpoint_dir="./checkpoints"):
    """
    PGD-based adversarial training (Madry et al., 2018).
    """
    model = model.to(device)
    criterion = nn.CrossEntropyLoss()
    optimizer = optim.SGD(
        model.parameters(),
        lr=config["learning_rate"],
        momentum=config["momentum"],
        weight_decay=config["weight_decay"])
    scheduler = CosineAnnealingLR(
        optimizer, T_max=config["epochs"])
    best_accuracy = 0.0

    for epoch in range(1, config["epochs"] + 1):
        model.train()
        train_loss, correct, total = 0.0, 0, 0

        for images, labels in train_loader:
            images = images.to(device)
            labels = labels.to(device)

            # Generate adversarial examples
            model.eval()  # BN in eval mode for attack
            adv_images = pgd_attack(
                model, images, labels,
                epsilon=adv_config["epsilon"],
                alpha=adv_config["alpha"],
                steps=adv_config["steps"],
                random_start=True, device=device)
            model.train()  # Back to train mode

            # Train on adversarial examples
            optimizer.zero_grad()
            outputs = model(adv_images)
            loss = criterion(outputs, labels)
            loss.backward()
            optimizer.step()

            train_loss += loss.item()
            _, predicted = outputs.max(1)
            total += labels.size(0)
            correct += predicted.eq(labels).sum().item()

        scheduler.step()

        # Evaluate and save best checkpoint
        pgd_acc = evaluate_pgd(model, test_loader,
                               adv_config, device)
        if pgd_acc > best_accuracy:
            best_accuracy = pgd_acc
            torch.save({
                "model_state_dict": model.state_dict(),
                "epoch": epoch,
                "pgd_accuracy": pgd_acc,
            }, f"{checkpoint_dir}/robust_best.pt")

    return model
\end{lstlisting}

\section{Evaluation Utilities}

\begin{lstlisting}[language=Python, caption={Robustness Evaluation}, label={lst:eval}]
import torch


def evaluate_robustness(model, test_loader, attack_fn,
                        epsilon_values, device):
    """
    Evaluate model robustness across epsilon values.

    Args:
        model: Neural network model
        test_loader: Test data loader
        attack_fn: Attack function
        epsilon_values: List of epsilon values
        device: Device to run on

    Returns:
        Dictionary of accuracies for each epsilon
    """
    model.eval()
    results = {}

    for epsilon in epsilon_values:
        correct = 0
        total = 0

        for images, labels in test_loader:
            images = images.to(device)
            labels = labels.to(device)

            adv_images = attack_fn(
                model, images, labels, epsilon,
                device=device)

            with torch.no_grad():
                outputs = model(adv_images)
                _, predicted = outputs.max(1)
                correct += (
                    predicted.eq(labels).sum().item())
                total += labels.size(0)

        results[epsilon] = 100.0 * correct / total

    return results
\end{lstlisting}

\end{document}
